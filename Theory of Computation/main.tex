\documentclass[a4paper, 12pt, margin=1.2in]{article}
\usepackage[margin=1.15in]{geometry}

\def \department {CSE}
\def \coursenum {105}
\def \nterm {Winter}
\def \nyear {2023}
\def\ncourse {Theory of Computation}

% HI if ur reading this, I'm updating all my notes Thursday/Friday, recovering / catching up from retreat

\makeatletter
\usepackage{alltt}
\usepackage{amsfonts}
\usepackage{amsmath}
\usepackage{amssymb}
\usepackage{amsthm}
\usepackage{babel}
\usepackage{booktabs}
\usepackage{caption}
\usepackage{enumitem}
\usepackage{fancyhdr}
\usepackage{graphicx}
\usepackage{mathdots}
\usepackage{mathtools}
\usepackage{microtype}
\usepackage{multirow}
\usepackage{pdflscape}
\usepackage{pgfplots}
\usepackage{siunitx}
\usepackage{slashed}
\usepackage{tabularx}
\usepackage{tikz}
\usepackage{tkz-euclide}
\usepackage[normalem]{ulem}
\usepackage[all]{xy}
\usepackage{imakeidx}

\ifx \nauthor\undefined
    \def\nauthor{Nolan Chai}
\else
\fi

\author{Notes taken by \nauthor}
\date{\nterm\ \nyear}

\makeindex[intoc, title=Index]
\indexsetup{othercode={\lhead{\emph{Index}}}}

\newcommand\clean[1]{{\usefont{T1}{clean}{m}{n} #1 }}

\ifx \nextra \undefined
    \usepackage[
        pdftex,
        hidelinks,
        pdfauthor={Nolan Chai},
        pdfsubject = {UCSD Notes: \department\ \coursenum\ - \ncourse},
        pdftitle={\department\ \coursenum\ - \ncourse},
        pdfkeywords = {UCSD UC San Diego \department\ \coursenum\ \nterm\ \nyear\ \ncourse}
    ] {hyperref}
    \title{\department\ \coursenum\ --- \ncourse}
\else
    \usepackage[
        pdftex,
        hidelinks,
        pdfauthor = {Nolan Chai},
        pdfsubject = {UCSD Notes: \department\ \coursenum\ - \ncourse},
        pdftitle = {\department\ \coursenum\ (\nextra)},
        pdfkeywords = {UCSD UC San Diego \department\ \coursenum\ \nterm\ \nyear\ \ncourse\ \nextra}
    ] {hyperref}
    \title{\department\ \coursenum\ --- \ncourse \\ {\Large \nextra}}
    \renewcommand\printindex{}
\fi
\pgfplotsset{compat=1.12}

\pagestyle{fancyplain}
\ifx \ncoursehead \undefined
\def\ncoursehead{\ncourse}
\fi

\lhead{\emph{\nouppercase{\leftmark}}}
\ifx \nextra \undefined
  \rhead{
    \ifnum\thepage=1
    \else
      \ncoursehead
    \fi}
\else
  \rhead{
    \ifnum\thepage=1
    \else
      \ncoursehead \ (\nextra)
    \fi}
\fi
\usetikzlibrary{arrows.meta}
\usetikzlibrary{decorations.markings}
\usetikzlibrary{decorations.pathmorphing}
\usetikzlibrary{positioning}
\usetikzlibrary{fadings}
\usetikzlibrary{intersections}
\usetikzlibrary{cd}

\makeatother

\begin{document}

\maketitle
\newpage
\tableofcontents

\newpage
\section*{\centering{Preface}}
These are a collection of notes personally taken by me, specifically for readings and allotted content for UCSD's CSE 105 Theory of Computation taken in Winter 2023. \\
These notes are not endorsed by the lecturers nor staff, and I have modified them (often significantly) over random periods of time. They may become nowhere near accurate representations of what was actually lectured, or written in the books, and are simply to aid in my own understanding. In particular, all errors are almost surely mine.\\
\textit{Notes will be reviewed, updated, and revised within 48 hours of each lecture.}
\newpage

\section*{Introduction}
Course covered by Miles Jones, more geared towards theory and mathematics rather than programming. In general, the major course topics and contents covered by these notes will include finite automata, pushdown automata, and Turing machines.

\section*{Logistics}
All lectures will be posted before class on the content calendar tab of the canvas site (also podcasted / available via Canvas).\\ 
There will be \textbf{short review quizzes due every Sunday at midnight}, with a total of 8. 
\subsection*{Exams}
\begin{itemize}
    \item Two tests on the Fridays of Week 4 and 8, being asynchronous.
    \item One midterm on Thursday of Week 6
    \item One Final Exam
\end{itemize} 
For each exam, you have the option of taking a retest with unlimited atttempts, potentially earning 1/3rd of the points you missed on the exam.
Official exam score 
\[
    \text{MAX} (\text{Test}, 2*\text{Test} + \text{ReTest})/3
\]
Discussions are not mandatory, but highly recommended, and are on Wednesdays 2-2:50 PM. \\
\textit{Office hours will become essential at one point.} Office hours include both regular office hours and 1-1s that you can sign up for. Office hours are usually for what's coming up, while 1-1s are for catching up.  \\
Three Grading Options: \\
\textbf{Option 1:}\\
Review Quizzes: 5\%, Homework: 30\%, Each Exam: 10\%, Midterm: 15\%, Final: 30\% \\
\textbf{Option 2:}\\
Review Quizzes: 5\%, Homework: 30\%, Each Exam: 10\%, Midterm: 0\%, Final: 45\% \\ 
\textbf{Option 3:}\\
Review Quizzes: 5\%, Homework: 30\%, Each Exam: 10\%, Midterm: 15\%, Final: 40\%\\ 
Jflap and flap.js are useful for drawing automata. Also \url{https://www3.nd.edu/~kogge/courses/cse30151-fa17/Public/other/tikz_tutorial.pdf} to draw automata. 

\newpage
\section{Lecture 1}

\subsection{CSE 105's Big Questions}
What problems are computers capable of solving?
What resources are needed to solve a problem?
Are some problems harder than others? 

\subsubsection{Computational Problems}
\begin{itemize}
    \item Find a file on your computer
    \item Determine if your code will compile
    \item Find a run-time error in your code
    \item Certify that your system is un-hackable
\end{itemize}
Which one of these problems is the hardest? \\We can quantify how hard an operation is - referred to as the 'hardness' - based on how many resources it uses, which is part of what we will focus on in this class.
\subsubsection{Computation Models}
A computational model has the following structure: You input something, like a string, and it will output a yes or a no in response. More specifically, our output will be based on whether or not the input string matches a specific pattern. Assume that whatever input you have has already been encoded.
\subsubsection{Vocabulary Review}
\begin{itemize}
    \item Set: a collection of "objects"
    \item Character/Symbol: a, b, c, 0, 1, @, ...
    \item Alphabet: A non-empty set of symbols (usually denoted with $\Sigma$).
    \item Word/String: A sequence of symbols
        \begin{itemize}
            \item 00010110
            \item Aazya00
        \end{itemize}
    \item Language: A set of strings
\end{itemize}
\textbf{Examples:}\\
\begin{tabular}{c|c}
     $\{ a,b,c,d,e \}$ & The set whose elements are a,b,c,d,e \\
    $| ababab | = 6$ & The length of the string ababab is 6 \\
    $| { a,b,c,d,e } | = 5$ & The size/cardinality of the set ${a,b,c,d,e}$ is 5
\end{tabular}
\subsection{Language Operations}
\subsubsection{Union}
Let $r_1$ and $r_2$ be languages over the alphabet $\Sigma$. \\
$r_1 \cup r_2$ is the language that includes any string that is either in $r_1$ or $r_2$.\\
Formally, this can be represented as:
\[
    r_1 \cup r_2 = \{ w | w \in r_1 \lor w \in r_2\}
\]
\textbf{Example.} What is $\{aa, ab, ba, a\} \cup \{bb, b, ba, \epsilon\}$ ?
\begin{align*}
    \{aa, ab, ba, a\} \cup \{bb, b, ba, \epsilon\} \\
    = |\{aa, ab, ba, a, bb, b, ba, \epsilon\}| \\
    = |\{aa, ab, ba, a, bb, b, ba\}|\\
    = 7
\end{align*}
\subsubsection{Concatenate}
Let $r_1$ and $r_2$ be languages over the alphabet $\Sigma$. \\
$r_1 \circ r_2$ is the language that includes any string from $r_1$ concatenated with any string from $r_2$.\\
Formally, this can be represented as:
\[
    r_1 \circ r_2 = \{w_1 \circ w_2 | w_1 \in r_1 \land w_2 \in r_2\}
\]
\textbf{Example.} What is $\{ab, ba, a\} \circ \{bb, b, \epsilon\}$? (Reminder that $\Sigma = \{a, b\}$
\begin{align*}
    \{ab, ba, a\} \circ \{bb, b, \epsilon\}
    = |\{abbb, abb, ab, babb, bab, ba, a\}|
    = 7
\end{align*}
\subsubsection{Kleene Star}
The Kleene star is made up of all concatentations within itself. Formally, the Kleene Star on the set S, $S^*$ is the set of all finite strings made up of concatenations of elements of S. \\
Given the set $S=\{a, b\}$, ${a, b}^*$ would include the following:
\begin{itemize}
    \item The empty string $\epsilon$
    \item a, aa, aaa
    \item b, bb, bbb
    \item ab, ababab, aaaaaaabbb, and so forth
\end{itemize}
The set would have infinitely many different elements (for most cases), but does \textbf{not} include infinite sequences of a's and b's (i.e., $a^\infty$, $b^\infty$).\\
Formally, this can be represented as:
\[
    S^* = \{w_1 \circ w_2 \circ ... \circ w_n | w_1, w_2, ..., w_n \in S \land n \geq 0 \}
\]
\textbf{For two cases, the Kleene star could produce a finite language}. When the language is empty, and a language with just an empty string g will produce a finite language instead of an infinite language. 
\subsection{Regular Expressions}
Regular expressions give us syntax to describe languages. For all isntances, suppose that the Alphabet here is $\{0, 1\}$. \\
The basic symbols in regular expressions are: $\emptyset, \epsilon, 0, 1$
\begin{itemize}
    \item $\emptyset$ means $\emptyset$
    \item $\epsilon$ means $\{\epsilon\}$
    \item $0$ means $\{0\}$
    \item $1$ means $\{1\}$
\end{itemize}
Basic operations include: $\cup$, $circ$, $*$, with parentheses to group.\\
Note: Often, $\circ$ is not used. We'd often rather write aab rather than $a \circ a \circ b$, or $(a \cup b) (c \cup b)$ rather than $(a \cup b) \circ (c \cup b)$. \\ \\
\textbf{Example.} How would you write these languages using regular expressions? 
\begin{itemize}
    \item $\{a, b, c\} = L(a \cup b \cup c)$
    \item $\{a^n | n \geq 0\} = \{\epsilon, a, aa, aaa, ...\} = L(a^*) = L(aa^*)$
    \item $\{a^n | n \geq 1\} = \{a, aa, aaa...\} = L(a^* \circ a) = L(a^*a)$
    \item $\{a, ab, abb, abbb, b, bb, bbb\} = L(a \cup ab \cup ...) = L((a \cup \epsilon) \circ (b \cup bb \cup bbb) \cup a)$
\end{itemize}

\newpage
\section{Lecture 2}
\subsection{Regular Expressions (Cont.)}
A regular expression is a string of symbols that \textit{describes} a language. You can think of stuff like single characters, empty string, the empty set, as atomic elements, strung together and recursively built up using our main three operators - union, concatenation, and Kleene star. Formally, if $R_1$ and $R_2$ are regular expressions, then
\begin{itemize}
    \item $R_1 \cup R_2$ is a regular expression that \textit{describes} the union of the languages described by $R_1$, $R_2$.
    \item $R_1 \circ R_2$ (or more commonly written as $R_1R_2$ is a regular expression that \textit{describes} the concatenation of the languages described by $R_1$, $R_2$.
    \item $R_1^*$ is a regular expression that \textit{describes} the Kleene star of the language described by $R_1$.
\end{itemize}
R is defined as a regular expression over the alphabet $\Sigma$ if
\begin{enumerate}
    \item $R = a$, where $a \in \Sigma$
    \item $R = \epsilon$
    \item $R = \emptyset$
    \item $R = (R_1 \cup R_2)$, where $R_1, R_2$ are themselves regular expressions.
    \item $R = (R_1 \circ R_2)$, where $R_1, R_2$ are themselves regular expressions.
    \item $(R_1^*),$ where $R_1$ is a regular expression.
\end{enumerate}
\textit{Notes: $\Sigma$ is shorthand for $(0 \cup 1)$ if $\Sigma = {0, 1}$. Parentheses may be omitted. $R^k$ means $R$ concatenated with itself $k$ times. } \\ \\
So one of our first big questions we're going to try to tackle is that:\\ We know how to build these regular expressions. We've seen that they can describe languages. We've seen some languages that they can describe. How far can we push it? What are the languages that we can describe by regular expressions? Can you describe it? Any languages? Any language? Are there languages out there that are too complicated to be described by regular expressions? So kinda keep this in your mind as we go. \\ \\
\textbf{Example.}
Let's say we have the set $A = \{1, 01, 001, ...\}$.\\ Then $0^*1$ is the \textit{regular expression} that describes $A$. It is clear that $0^*1$ is a regular expression because $*$ only acts on sets. However, we \textbf{DO NOT SAY THAT} $0^*1 = A$. \\Instead, we say that $\boxed{L(0^*1) = A}$, where $L$ stands for \textit{"The Language of..."}. \\ \\
\textit{Note: the author of the \textit{Theory of Computation} textbook that we are using doesn't do this and is a bit more loose with their writing. However, we will be more stringent with this kind of notation. } \\ 
\\
Furthermore, there is an order of operations to the regular expressions described. 
\begin{enumerate}
    \item Kleene Star
    \item Concatenation
    \item Union
\end{enumerate}
\textbf{Example.} So if we had the following two sets $B$ and $C$, what can you say about them? Are they the same or different? If they are different, then how do they differ? 
\[
    \text{Let }B = L(00^*1) \text{ and let } C = L((00)^*1)
\]
If we were to expand these expressions, we'd have:
\[B = \{01, 001, 0001, 00001, ...\}, C = \{1, 001, 00001, 0000001\}\]
As you can see, the string $1$ is contained in $C$, but not contained in $B$, and some strings from $B$ are not contained in $C$.
\subsection{The Alphabet}
Often, we will use the symbol $\Sigma$ to be the alphabet that we are using. \textit{This can differ from problem to problem}, so we will define it each time. \\ \\
The most common alphabets you will see in this class are: $\{0, 1\}$ or $\{a, b\}$. If $\Sigma$ is an alphabet, then $\Sigma^*$ is the set of all finite strings using symbols from the alphabet $\Sigma$. \\ \\
\textbf{Exercise.} If $\Sigma = \{1, a, \$\}$, then what is $\Sigma^*$? \\
\[
    Sigma^* = \{\epsilon, 1, a, \$, a\$, ...\}
\]
\newpage
\section{Review Quiz 1}
Highly recommended to complete the quiz before reading.

\end{document}